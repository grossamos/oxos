\documentclass{scrartcl}
\usepackage{fontenc}

\title{Dispositionspapier zur Studienarbeit\\Implementierung einer Kernel für Tic-tac-toe}
\author{ Amos Groß }
\date{13.10.2023}

\begin{document}

\maketitle

\section{Kurzbeschreibung der Arbeit}
Worum geht es in der Arbeit? 
% In dieser Arbeit soll eine Betriebssystem-Kernel entwickelt werden, welche die Grundlegenden Fähigkeiten besitzt um das Spiel "Tic-tac-toe" zu spielen.
% Die Kernel soll in Rust geschrieben werden und auf einem Raspberry Pi lauffähig sein.
Wie ist die (aktuelle) Ausgangssituation? 
% Die Kernel soll von Grund auf implementiert werden.
% Ausgang des Projektes ist daher keine bestehende Code-Basis.

% Notes:
% Habe einen Raspberry pi + knöpfe für GPIO eingaben auf dem das betriebssytem laufen soll.
% Es gibt keine bestehende code basis auf der das betriebssytem aufbaut

Welches Themenfeld wird bearbeitet? 
% Notes:
% Betriebssysteme und Rechnerarchitekturen

Welche Problemstellung soll angegangen werden? 
% IDK, schauen ob auf dem betriebssytem tic tac toe laufen kann?

Welche Grundlagen müssen vorhanden sein und welche Randbedingungen sind gegeben? 
% Grundlagen um ARM, Betriebssysteme

Welche Zielsetzungen gibt es in dieser Arbeit? 
Welche methodische Vorgehensweise wird gewählt?  

Dies soll möglichst in einem Fließtext dokumentiert werden. Idealerweise abschließend mit sehr konkreten Zielbeschreibungen, die auch validierbar sind.


\section{Gliederung und Zeitplan}
<Identifikation der wesentlichen Arbeitsschritte. Meilensteinplan. Konsequenzen
und Möglichkeiten der Meilensteine. Zeitplan bis zur Beendigung des praktischen
Teils sowie der Dokumentation.> <Eine erste Gliederung der Arbeit. Benennung von
Kapiteln und Unterkapiteln. Dies gilt als Leitfaden, noch nicht als abschließend.>

\section{Grundlegende Literatur}
<Belegen der Ausgangssituation. Wer hat auf ähnlichem Themenfeld bereits
gearbeitet? Wie passt die Studienarbeit in die aktuelle wissenschaftliche
Landschaft und was ist neu (dies wird oben dargelegt und hier belegt). Was wird
durch die erstellte Lösung verbessert und wie wird dies nachgewiesen?>

\end{document}
